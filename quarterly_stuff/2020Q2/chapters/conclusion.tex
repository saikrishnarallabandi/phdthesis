\chapter{Conclusion}

In this thesis, I present an argument for De-Entanglement: a property that has potential to isolate the factors of variation in the data distribution. I am interested in knowing if explicitly isolating relevant factors using such an approach is helpful with respect to downstream tasks. I first highlight three different approaches to accomplish `De-Entanglement'.  I then present one case study per approach to investigate the importance of such an approach. I conclude by arguing that while this serves as a neat framework to build systems, such an approach might not always be applicable or necessary.